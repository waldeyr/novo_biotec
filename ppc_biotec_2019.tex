%!TEX program = pdflatex
%%%%%%%%%%%%%%%%%%%%%%%%%%%%%%%%%%%%%%%%%
% The Legrand Orange Book
% LaTeX Template
% Version 1.4 (12/4/14)
%
% This template has been downloaded from:
% http://www.LaTeXTemplates.com
%
% Original author:
% Mathias Legrand (legrand.mathias@gmail.com)
%
% License:
% CC BY-NC-SA 3.0 (http://creativecommons.org/licenses/by-nc-sa/3.0/)
%
% Compiling this template:
% This template uses biber for its bibliography and makeindex for its index.
% When you first open the template, compile it from the command line with the 
% commands below to make sure your LaTeX distribution is configured correctly:
%
% 1) pdflatex main
% 2) makeindex main.idx -s StyleInd.ist
% 3) biber main
% 4) pdflatex main x 2
%
% After this, when you wish to update the bibliography/index use the appropriate
% command above and make sure to compile with pdflatex several times 
% afterwards to propagate your changes to the document.
%
% This template also uses a number of packages which may need to be
% updated to the newest versions for the template to compile. It is strongly
% recommended you update your LaTeX distribution if you have any
% compilation errors.
%
% Important note:
% Chapter heading images should have a 2:1 width:height ratio,
% e.g. 920px width and 460px height.
%
%%%%%%%%%%%%%%%%%%%%%%%%%%%%%%%%%%%%%%%%%

%----------------------------------------------------------------------------------------
%	PACKAGES AND OTHER DOCUMENT CONFIGURATIONS
%----------------------------------------------------------------------------------------

\documentclass[11pt,fleqn]{book} % Default font size and left-justified equations

\usepackage[top=3cm,bottom=3cm,left=3.2cm,right=3.2cm,headsep=10pt,a4paper]{geometry} % Page margins

\usepackage[table,xcdraw]{xcolor} % Required for specifying colors by name
\definecolor{verde}{RGB}{51,153,51} % Define the color used for highlighting throughout the book
\definecolor{blue}{rgb}{0.2, 0.2, 0.6}
\definecolor{red}{rgb}{0.8, 0.0, 0.0}
\definecolor{green}{rgb}{0.0, 0.42, 0.24}

% Font Settings
\usepackage{avant} % Use the Avantgarde font for headings
%\usepackage{times} % Use the Times font for headings
\usepackage{mathptmx} % Use the Adobe Times Roman as the default text font together with math symbols from the Sym­bol, Chancery and Com­puter Modern fonts

\usepackage{microtype} % Slightly tweak font spacing for aesthetics
\usepackage[utf8]{inputenc} % Required for including letters with accents
\usepackage[T1]{fontenc} % Use 8-bit encoding that has 256 glyphs
\hyphenation{Mi-nis-té-ri-o}


% Index
\usepackage{calc} % For simpler calculation - used for spacing the index letter headings correctly
\usepackage{makeidx} % Required to make an index
\makeindex % Tells LaTeX to create the files required for indexing
\usepackage{verbatim}

\usepackage[colorinlistoftodos,prependcaption,textsize=tiny,linecolor=red,backgroundcolor=red!25,bordercolor=red]{todonotes}
\usepackage{epigraph}
\renewcommand{\textflush}{flushepinormal}
\setlength{\epigraphwidth}{0.8\textwidth}

\usepackage{nameref}
\usepackage{booktabs}
\usepackage{graphicx}
\usepackage{float}
\usepackage{multirow}

\usepackage[normalem]{ulem}%tachado


% Bibliography
%\usepackage[backend=biber,style=authoryear,autocite=inline, citestyle=authoryear]{biblatex}
\usepackage[style=abnt]{biblatex}
\addbibresource{bibliography.bib} % BibTeX bibliography file
\defbibheading{bibempty}{}
\renewcommand*{\nameyeardelim}{\addcomma\space}

\newcommand{\VER}[1]{\begingroup\color{red}#1\endgroup}

%----------------------------------------------------------------------------------------

\input{structure} % Insert the commands.tex file which contains the majority of the structure behind the template

\begin{document}

\let\cleardoublepage\clearpage

\renewcommand{\chaptername}{Capítulo}
\renewcommand{\figurename}{Fig.}

%----------------------------------------------------------------------------------------
%	TITLE PAGE
%----------------------------------------------------------------------------------------
\begingroup
	\thispagestyle{empty}
	
	\AddToShipoutPicture*{\put(0,0){\includegraphics[scale=1]{capa}}} % Image background
	
	\AddToShipoutPicture*{\put(116,650){\includegraphics[scale=.75]{brasao.png}}} % Image background
	
	\AddToShipoutPicture*{\put(244,200){\includegraphics[scale=0.2]{ifgvertical}}} % Image background
	
	\vspace*{4.5cm}
	
	\centering
	\par
	{\Huge Projeto Pedagógico}\vspace*{1.5cm}
	\par
	\fontsize{40}{40}
	\selectfont
	Técnico Integrado em Tempo Integral em Biotecnologia\vspace*{10cm}
	\par
	{\Huge 2019}
	\par
\endgroup
\pagebreak

%----------------------------------------------------------------------------------------
%	PEOPLE PAGE
%----------------------------------------------------------------------------------------
\chapterimage{banner3} % Chapter heading image
\begin{center}
	\par
	{\large PRESIDENTE DA REPÚBLICA \\ Jair Messias Bolsonaro}\vspace*{1cm}
	\par
	{\large MINISTRO DA EDUCAÇÃO \\ Nome do Ministro}\vspace*{1cm}
	\par
	{\large SECRETÁRIO DE EDUCAÇÃO PROFISSIONAL E TECNOLÓGICA \\ Alexandro Ferreira de Souza}\vspace*{1cm}
	\par
	{\large REITOR DO INSTITUTO FEDERAL DE GOIÁS \\ Jerônimo Rodrigues da Silva}\vspace*{1cm}
	\par
	{\large PRÓ-REITOR DE PESQUISA E PÓS-GRADUAÇÃO \\ Ruberley Rodrigues de Souza}\vspace*{1cm}
	\par
	{\large DIRETORIA DE PÓS-GRADUAÇÃO \\ Clarinda Aparecida da Silva}\vspace*{1cm}
	\par
	{\large COORDENADOR DO CURSO \\ Waldeyr Mendes Cordeiro da Silva}\vspace*{1cm}
\end{center}

\chapterimage{banner3} % Chapter heading image
\renewcommand\contentsname{Sumário}
\tableofcontents

%----------------------------------------------------------------------------------------
%	CHAPTER
%----------------------------------------------------------------------------------------
\chapterimage{01.jpg} % Chapter heading image
\chapter{Apresentação}
\vspace{6em}
\begin{flushright}
	\textit{ }
\end{flushright}
\vspace{12em}
\indent


\section{Identificação do Curso}
\begin{itemize}
	\item \textbf{Instituição Proponente:} Instituto Federal de Educação, Ciência e Tecnologia de Goiás
	\item \textbf{Nome do curso:} Técnico Integrado em Tempo Integral em Biotecnologia
	\item \textbf{Carga Horária do Curso:} 3802 horas
	\item \textbf{Forma de oferta:} Presencial
	\item \textbf{Duração:} 3 anos
	\item \textbf{Número de Vagas:} 30 vagas anuais
	\item \textbf{Local de Oferta:} Instituto Federal de Goiás - Câmpus Formosa
	\item \textbf{Reitor:} Jerônimo Rodrigues da Silva
	\item \textbf{Pró-Reitora de Ensino:} Oneida Cristina Gomes Barcelos Irigon
	\item \textbf{Pró-Reitor de Pesquisa, Pós-Graduação e Inovação:} Ruberley Rodrigues de Souza
	\item \textbf{Diretoria de Pós-Graduação:} Clarinda Aparecida da Silva
\end{itemize}

%-----------------------------------------------
\newpage  
%------------------------------------------------
\section{Comissão Organizadora}
	\begin{itemize}[label=\bfseries]
		\item \nameref{AdrianoDarosci}
		\item ...
		\item \nameref{WaldeyrMendes}
	\end{itemize}

%----------------------------------------------------------------------------------------
%	CHAPTER
%----------------------------------------------------------------------------------------
\chapterimage{02.jpg} % Chapter heading image

\chapter{Introdução}
\vspace{6em}
\begin{flushright}
	\textit{\textcolor{white}{Foto: Adriano Darosci}}
\end{flushright}
\vspace{12em}
\indent

De acordo com a Convenção sobre Diversidade Biológica da ONU em 1992, biotecnologia é qualquer aplicação tecnológica que use sistemas biológicos, organismos vivos ou derivados destes, para fazer ou modificar produtos ou processos para usos específicos.
A biotecnologia abrange processos microbiológicos, organismos vivos e biossistemas para produzir novas práticas e produtos.
Em todo o mundo, incluindo o Brasil, a biotecnologia vêm permeando, modificando e impulsionando inúmeras áreas da economia~\cite{dias2017bioeconomiabrasil}, incluindo a bioeconomia, que pode ser entendida como a economia baseada em materiais, químicos e energia derivada de fontes biológicas renováveis~\cite{mccormick2013bioeconomy}.
O desenvolvimento da bioeconomia depende de desenvolver capacidades para explorar a biodiversidade. 
Entretanto, a sustentabilidade não é uma característica inerente da bioeconomia~\cite{pfau2014visions}, o que traz a necessidade de pesquisa em abordagens que busquem explorar o potencial biotecnológico de forma integrada com sua preservação.

O Projeto Pedagógico do Curso – PPC está organizado a partir dos Eixos Tecnológicos constantes do Catálogo Nacional dos Cursos Técnicos - CNTC~\cite{CatalogoCursosTecnicos2016}.
Apesar de ligado ao Eixo de Produção Industrial, é um curso altamente interdisciplinar com possibilidades de verticalização para cursos de graduação no itinerário formativo superior de tecnologia em biotecnologia, superior de tecnologia em saneamento ambiental, bacharelado ou licenciatura em ciências biológicas, bacharelados em biomedicina, farmácia, nutrição, em engenharia de alimento, em engenharia química, em biotecnologia e em engenharia ambiental.

\section{Justificativa}
\indent

Desenvolver a educação profissional e tecnológica como processo educativo e investigativo de geração e adaptação de soluções técnicas e tecnológicas às demandas sociais e peculiaridades regionais está entre as finalidades e características dos Institutos Federais~\cite{Lei11892De2008}. 
A Lei 11.892 de 2008~\cite{Lei11892De2008} indica ainda como objetivo que os Institutos Federais devem ministrar educação profissional técnica de nível médio, prioritariamente na forma de cursos integrados ao Ensino Médio.

Dessa forma, alinhado às demandas regionais e nacionais, o IFG câmpus Formosa, vem ofertando o curso de Biotecnologia desde 2011 com uma eficiência acadêmica superior a 86\%.
O curso já passou por uma atualização em 2014 e este projeto pedagógico traz sua segunda atualização, com vistas a adequar-se simultaneamente às recentes mudanças promovidas pela reforma do Ensino Médio, ao mesmo tempo que busca manutenir e melhorar seus números de eficiência acadêmica tal como definido em Lei~\cite{Lei13005De2014} e no Plano de Desenvolvimento Institucional 2019-2013 do IFG~\cite{Resolucao32De2018}.

Além dos aspectos legais, esta atualização promove um novo olhar pedagógico na organização didática das disciplinas.
Tais mudanças, pautadas no perfil do egresso do curso, levam em conta sua formação integral afim de promover o desenvolvimento do estudante na sua totalidade, considerando as dimensões físico-psíquico-cognitiva, histórica, social e profissional.


Em relação a aspectos sócio-econômicos, a cidade de Formosa conta com uma população urbana de 106.462 habitantes de acordo com a PMAD – Pesquisa Metropolitana por Amostra de Domicílios~\cite{pmad2017codeplan}, onde cerca de 40\% são jovens até 24 anos, 20\% dos quais entre 15 e 24 anos. 
Essa faixa da população é o público alvo deste curso Técnico Integrado ao Ensino Médio juntamente com os demais cursos superiores do IFG.

\subsection{Objetivos}
Em consonância com as Bases Curriculares Nacionais, e Diretrizes para o Ensino Médio Profissionalizante, os objetivos do curso são:
\begin{itemize}
\item Atender às expectativas dos estudantes e às demandas da sociedade contemporânea para a formação no Ensino Médio e Profissional
\item Garantir aos estudantes o protagonismo no processo de escolarização, reconhecendo-os como interlocutores legítimos sobre currículo, ensino e aprendizagem
\item Proporcionar experiências e processos que lhes garantam as aprendizagens necessárias para a leitura da realidade, o enfrentamento dos novos desafios da contemporaneidade (sociais, econômicos e ambientais) e a tomada de decisões éticas e fundamentadas
\end{itemize}

\section{Requisitos para Acesso ao Curso}

\todo[inline]{Texto...}


\section{Perfil do Egresso}
\indent

O Técnico em Biotecnologia executa atividades laboratoriais de biotecnologia e biociências em centros de pesquisas, indústrias e empresas no setor de saúde humana e animal, ambiental e agropecuário. 
Opera, controla e monitora processos industriais e laboratoriais, incluindo laboratórios de saúde e ambiental. 
Prepara materiais, meios de cultura, soluções e reagentes. 
Analisa substâncias e materiais biológicos. 
Cultiva \textit{in vivo} e \textit{in vitro} microrganismos, células e tecidos animais e vegetais. 
Realiza o preparo de amostras dos tecidos animais e vegetais. 
Extrai, replica e quantifica biomoléculas.
Realiza a produção de imunobiológicos, vacinas, diluentes, kits de diagnóstico e bioprocessos industriais. 
Colabora nas atividades de perícia criminal e investigação genética. 
Desenvolve pesquisa de melhoramento genético. 
Opera a criação e manejo de animais de experimentação. 
Controla a qualidade e a compra de matérias-primas, insumos e produtos.

\subsection{Campo de atuação}
\indent

O campo de atuação inclui, mas não limita-se a empresas, indústrias, agroindústrias, instituições de pesquisa, ensino e desenvolvimento em biociências e produtos biotecnológicos.
Laboratórios de controle de qualidade de biomoléculas, de bioprocessos, de biologia molecular, de toxicologia, de biodiagnósticos e de análises clínicas. 
Bancos de materiais biológicos e de genes. 
Empresas de consultorias, assistência técnica, comercialização de insumos e equipamentos utilizados na área de biociências e biotecnologia. 
Indústrias alimentícias, de cosméticos, bebidas e farmacêutica. 
Laboratório de agropecuária e ambiental. 
Estações de monitoramento e tratamento biológicos da água. 
Escritórios de patentes biotecnológicas. 
Empreendimento próprio.

\subsection{Ocupações CBO associadas}
\indent

325305-Técnico em biotecnologia. 

325310-Técnico em imunobiológicos.

\subsection{Normas associadas ao exercício profissional}
\indent

Lei nº 11.105/2005. 

Decreto nº 5.591/2005. 

Decreto nº 5.705/2006. 

Decreto nº 5.950/2006. 

Decreto nº 6.041/2007.

Decreto nº 6.925/2009. 

\subsection{Possibilidades de verticalização para cursos de graduação no itinerário formativo}

Curso superior de tecnologia em biotecnologia. 
Curso superior de tecnologia em saneamento ambiental. 
Bacharelado em ciências biológicas. 
Bacharelado em biomedicina. 
Bacharelado em farmácia. 
Bacharelado em nutrição. 
Bacharelado em engenharia de alimentos. 
Bacharelado em engenharia química. 
Bacharelado em biotecnologia. 
Bacharelado em engenharia ambiental.

%----------------------------------------------------------------------------------------
%	CHAPTER
%----------------------------------------------------------------------------------------
\chapterimage{03.jpg} % Chapter heading image

\chapter{Organização do Curso}
\vspace{6em}
\begin{flushright}
	\textit{\textcolor{white}{``O homem, por ser inacabado, tende à perfeição. A educação é, portanto, um
			processo contínuo que só acaba com a morte'' (FURTER, 1973).}}
\end{flushright}
\vspace{12em}

\section{Forma de Oferta}\label{carga}
\indent

As Diretrizes Curriculares Nacionais para o Ensino Médio~\cite{Resolucao032018}, recentemente atualizadas, em consonância com a Lei de Diretrizes e Bases da Educação~\cite{Lei19394De1996}, trazem em seu Capítulo II as formas de Oferta e Organização Curricular. 
De acordo com as Diretrizes Curriculares Nacionais para o Ensino Médio, entre outros:

\begin{itemize}
	\item A oferta do Ensino Médio deve ser assegurada para todos os estudantes, sejam adolescentes,
	jovens ou adultos;
	\item O Ensino Médio pode organizar-se em tempos escolares de vários formatos:
	\begin{itemize}
		\item Séries anuais;
		\item Períodos Semestrais;
		\item Ciclos;
		\item Módulos;
		\item Sistema de Créditos;
		\item Alternância Regular de Períodos de estudos;
		\item Grupos não seriados com base na idade, competências e outros critérios;
	\end{itemize}
	\item O Ensino Médio deverá ter três anos no mínimo.
	\item A Educação a Distância pode ser usada para compor a carga horária do Ensino Médio Diurno na proporção de até 20\%;
\end{itemize}

Desta forma, o Curso Técnico Integrado ao Ensino Médio em Biotecnologia, em consonância com as mais recentes reestruturações do ensino, funcionará em período matutino e vespertino.

\subsection{Quantidade de Vagas}
\indent

30 (trinta) vagas anuais. 

\subsection{Duração e Carga Horária}
\indent

O Curso tem duração total de~\VER{3802} horas distribuídas em três anos.
A carga horária total está assim distruída:
\begin{itemize}
	\item \VER{3078} horas de disciplinas com ementas categóricas
	\item \VER{324} horas de disciplinas com ementas flexíveis
	\item \VER{200} horas de estágio curricular
	\item \VER{200} horas de atividades complementares
\end{itemize}
, sendo \VER{3402} horas de disciplinas e \VER{200} horas de estágio curricular supervisionado e \VER{200} horas de atividades complementares.
A duração mínima é de 3 (três) anos e o prazo máximo de integralização dos cursos da educação profissional técnica de nível médio integrado ao ensino médio é do dobro do tempo da sua duração. 
Logo, o máximo será de 6 (seis) anos, em conformidade com a legislação vigente. 
Após o prazo previsto por lei o aluno terá que se submeter a novo processo seletivo, caso deseje concluí-lo.


\section{Matriz Curricular}\label{matriz}
\indent

As disciplinas estão organizadas em séries anuais, conforme a Tabela~\ref{tab:matriz}, com a seguinte perspectiva:
\begin{itemize}
	\item 1º Ano: Construção das bases científicas e epistemológicas; Desenvolvimento das competências previstas na Base Nacional Comum Curricular (BNCC);
	\item 2º Ano: Contextualização das bases científicas e epistemológicas; Desenvolvimento das competências previstas na Base Nacional Comum Curricular (BNCC); Desenvolvimento de competências primárias em Biotecnologia;
	\item 3º Ano: Consolidação e aplicação das bases científicas e epistemológicas; Desenvolvimento das competências previstas na Base Nacional Comum Curricular (BNCC); Desenvolvimento de competências profissionais em Biotecnologia;
\end{itemize}


\begin{table}[H]
	\centering
	\caption{Matriz Curricular do Curso Técnico em Biotecnologia Integrado ao Ensino Médio em Tempo Integral.}
	\label{tab:matriz}
	\resizebox{\textwidth}{!}{%
		\begin{tabular}{|l|l|l|l|l|l|}
			\hline
			Núcleo                                                                     & Disciplinas                                      & 1º Ano     & 2º Ano    & 3º Ano    & CH Totais \\ \hline
			\multirow{12}{*}{Básico}& \nameref{disc:artes}                             & 2          & 2         & 0         & 108       \\ \cline{2-6} 
			& \nameref{disc:ingles}                                                    & 4          & 0         & 0         & 108       \\ \cline{2-6} 
			& \nameref{disc:educacaofisica}                                            & 2          & 2         & 0         & 108       \\ \cline{2-6} 
			& \nameref{disc:linguaportuguesa}                                          & 4          & 4         & 4         & 324       \\ \cline{2-6} 
			& \nameref{disc:matematica}                                                & 2          & 4         & 2         & 216       \\ \cline{2-6} 
			& \nameref{disc:geografia}                                                 & 2          & 2         & 2         & 162       \\ \cline{2-6} 
			& \nameref{disc:historia}                                                  & 2          & 2         & 2         & 162       \\ \cline{2-6} 
			& \nameref{disc:fisica}                                                    & 2          & 2         & 2         & 162       \\ \cline{2-6} 
			& \nameref{disc:quimica}                                                   & 4          & 0         & 0         & 108       \\ \cline{2-6} 
			& \nameref{disc:biologia}                                                  & 4          & 0         & 0         & 108       \\ \cline{2-6} 
			& \nameref{disc:filosofia}                                                 & 2          & 2         & 2         & 162       \\ \cline{2-6} 
			& \nameref{disc:sociologia}                                                & 2          & 2         & 2         & 162       \\ \hline
			\multicolumn{2}{|l|}{Carga horária semanal Núcleo Básico}                  & 32         & 22        & 16        & 1890      \\ \hline
			\multirow{7}{*}{Politécnico}   
			& \nameref{disc:info}                                                      & 2          & 0         & 0         & 54        \\ \cline{2-6} 
			& \nameref{disc:espanhol_libras}                                           & 0          & 2         & 0         & 54        \\ \cline{2-6} 
			& \nameref{disc:fermentacao}                                               & 0          & 0         & 2         & 54        \\ \cline{2-6} 
			& \nameref{disc:bioquimica}                                                & 0          & 2         & 0         & 54        \\ \cline{2-6} 
			& \nameref{disc:microbiologia}                                             & 0          & 0         & 4         & 108       \\ \cline{2-6} 
			& \nameref{disc:biomol}                                                    & 0          & 4         & 0         & 108       \\ \cline{2-6} 
			& \nameref{disc:bioeticalab}                                               & 4          & 0         & 0         & 108       \\ \hline			
			\multicolumn{2}{|l|}{Carga horária semanal dos Núcleos Básico  e Politécnico}  & 38         & 30        & 22        & 2430      \\ \hline
			\multirow{7}{*}{Técnico}       
			& \nameref{disc:biotecAnimal}                                              & 0          & 2         & 0         & 54       \\ \cline{2-6} 
			& \nameref{disc:biotecVegetal}                                             & 0          & 2         & 0         & 54       \\ \cline{2-6} 
			& \nameref{disc:biotecAlimentos}                                           & 0          & 0         & 4         & 108       \\ \cline{2-6} 
			& \nameref{disc:biotecFarmacos}                                            & 0          & 0         & 4         & 108       \\ \cline{2-6} 
			& \nameref{disc:biotecSaude}                                               & 0          & 0         & 4         & 108       \\ \cline{2-6} 
			& \nameref{disc:producao}                                                  & 0          & 0         & 4         & 108       \\ \cline{2-6} 
			& \nameref{disc:analitica}                                                 & 0          & 4         & 0         & 108       \\ \hline
			\multicolumn{2}{|l|}{Carga horária semanal dos Núcleos Básico, Politécnico e Profissional} 
			                                                                           & 38         & 38        & 38        & 3078      \\ \hline
			\multicolumn{2}{|l|}
			{\nameref{disc:estudo}}                                                    & 4          & 4         & 4         & 324       \\ \hline
			\multicolumn{5}{|l|}{Estágio curricular}                                                                        & 200       \\ \hline
			\multicolumn{5}{|l|}{Atividades complementares}                                                                 & 200       \\ \hline
			\multicolumn{2}{|l|}{\multirow{2}{*}{Carga horária}}                       & \multicolumn{3}{l|}{Total Semanal} & Total     \\ \cline{3-6} 
			\multicolumn{2}{|l|}{}                                                     & 42         & 42        & 42        & 3802      \\ \hline
			\multicolumn{2}{|l|}{Número de disciplinas por ano}                        & 14         & 15        & 13        & 42        \\ \hline
		\end{tabular}%
	}
\end{table}

\section{Metodologia de Ensino-Aprendizagem}\label{metodologia}
\indent

\todo[inline]{Descrever a abordagem de Núcleo Politécnico, e curriculo integrado...}


\subsection{Trabalhos Discentes}\label{trabdiscentes}
\indent


\subsubsection{Atividades Não Presenciais}
\indent

Respeitados os mínimos previstos de duração e carga horária total, o plano de curso técnico de nível médio pode prever atividades não presenciais, até 20\% (vinte por cento) da carga horária diária do curso, desde que haja suporte tecnológico e seja garantido o atendimento por docentes e tutores~\cite{Resolucao06De2012}.

\subsubsection{Avaliação}

\todo[inline]{Utilizar a disciplina de debates semanais como 50\% da nota para todas as disciplinas?}

\section{Certificação}
\indent

Para obter o Certificado de Especialização em Ambiente, Ciência e Ensino do Cerrado, o discente deverá satisfazer as seguintes exigências:
\begin{itemize}
	\item Ser aprovado em todas as disciplinas do curso com nota mínima igual a 7,0 (sete) e freqüência igual ou superior a 75\% da carga horária da disciplina;
	\item Defender publicamente a monografia produzida perante uma Banca composta por, no mínimo, três professores (orientador, mais dois professores convidados, um externo e um interno ao campus IFG/Formosa) obtendo conceito Aprovado (AP);
	\item Possuir pelo menos um certificado que comprove a apresentação (pôster ou oral) de resultados relacionados à monografia exigida por essa pós-graduação em evento científico externo ao IFG (congressos, seminários, simpósios) cuja abrangência é, no mínimo, regional;
	\item Possuir documento (carta eletrônica ou impressa) que ateste a submissão em um periódico indexado de pelo menos um artigo produzido a partir dos resultados obtidos com o trabalho de conclusão de curso exigido por essa pós-graduação;
	\item Comprovar a quitação de suas obrigações junto à biblioteca do Campus Formosa do Instituto Federal de Educação, Ciência e Tecnologia de Goiás;
	\item Entregar toda a documentação exigida pelo processo seletivo.
\end{itemize}

O Certificado será emitido pelo Instituto Federal de Educação, Ciência e Tecnologia de Goiás, nos termos da Resolução CNE/CES n.º 1, de 8 de junho de 2007.	

%----------------------------------------------------------------------------------------
%	CHAPTER
%----------------------------------------------------------------------------------------
%------------------------------------------------
\chapterimage{04.jpg} % Chapter heading image

\chapter{Ementas}
\vspace{6em}
\begin{flushright}
	\textit{\textcolor{white}{Foto: Adriano Darosci}}
\end{flushright}
\vspace{12em}

\section{Ementas do Núcleo Básico}\label{ementasBasico}
\indent

O núcleo básico comtempla os conhecimentos e as habilidades nas áreas de linguagens e códigos, ciências humanas, matemática e ciências da natureza, vinculados à Educação Básica (que) deverão permear o currículo dos cursos técnicos de nível médio, de acordo com as especificidades dos mesmos, como elementos essenciais para a formação e o desenvolvimento profissional do cidadão~\cite{Resolucao06De2012}.

As disciplinas aqui dispostas, visam viabilizar o acesso dos estudantes às bases científicas e tecnológicas dos processos de produção do mundo contemporâneo, relacionando teoria e prática – ou o conhecimento teórico à resolução de problemas da realidade social, cultural ou natural~\cite{BNCC2019}.


\newpage
\subsection{Artes}\label{disc:artes}
\begin{itemize}
	\item \textbf{Carga horária (hora/aula):} 108
	\item \textbf{Docente Responsável:}
	\item \textbf{Conceitos-Chave:}
	\begin{itemize}
		\item NNNNN
		\item NNNNN
	\end{itemize}

	\item \textbf{Ementa 1º Ano:}
	Estudo sobre arte em suas linguagens, códigos e tecnologias específicas e suas influências culturais e educativas na sociedade;
	Conhecimento da arte como identidade, memória e criação, considerando suas expressões regionais e ressaltando as influências africanas e indígenas;
	Fundamentos, conceitos, funções, especificidades e características das artes visuais, dança, música, teatro e audiovisual;
	\item \textbf{Ementa 2º Ano:}
	Abordagens histórico-reflexivas das produções artístico-culturais da humanidade;
	Projetos de investigação e experimentação artística com técnicas, materiais, estilos e gêneros variados;
	Apreciação e compreensão de diferentes poéticas em diálogo com as manifestações artísticas regionais nas diversas linguagens;
	Estudo das matrizes culturais da arte brasileira, em especial as africanas e indígenas, a partir das diversas visões e versões de seus representantes;
	Relações entre arte e mundo do trabalho;
	\item \textbf{Bibliografia básica}
	\begin{enumerate}
		\item NNNNN
	\end{enumerate}
	\item \textbf{Bibliografia complementar}
	\begin{enumerate}
		\item 
	\end{enumerate}	
\end{itemize}

\newpage
\subsection{Língua estrangeira - Inglês}\label{disc:ingles}
\begin{itemize}
	\item \textbf{Carga horária (hora/aula):} 108
	\item \textbf{Docente Responsável:}
	\item \textbf{Conceitos-Chave:}
	\begin{itemize}
		\item NNNNN
		\item NNNNN
	\end{itemize}
	\item \textbf{Ementa 1º Ano:} 
	Leitura, compreensão e interpretação de textos orais e escritos, estabelecendo relações entre língua, cultura e sociedade;
	Estudo de elementos morfossintáticos, semânticos e fonológicos da língua inglesa;
	Desenvolvimento das habilidades comunicativas com ênfase na leitura;
	Leitura, compreensão e interpretação de textos escritos, ligados à área de conhecimento do curso;
	\item \textbf{Bibliografia básica}
	\begin{enumerate}
		\item NNNNN
	\end{enumerate}
	\item \textbf{Bibliografia complementar}
	\begin{enumerate}
		\item 
	\end{enumerate}	
\end{itemize}


\newpage
\subsection{Educação física}\label{disc:educacaofisica}
\begin{itemize}
	\item \textbf{Carga horária (hora/aula):} 108
	\item \textbf{Docente Responsável:}
	\item \textbf{Conceitos-Chave:}
	\begin{itemize}
		\item NNNNN
		\item NNNNN
	\end{itemize}
	\item \textbf{Ementa 1º Ano:} 
	Introdução ao estudo, vivência e reflexão crítica dos temas da cultura corporal de movimento, abordados pela Educação Física;
	Compreensão dos aspectos biológicos, históricos, psicológicos, sociais, filosóficos e culturais, e suas relações com o meio ambiente e a diversidade humana, em uma perspectiva omnilateral;

	\item \textbf{Ementa 2º Ano:} 
	Aprofundamento e ampliação do estudo, vivência e reflexão crítica dos temas da cultura corporal de movimento, abordados pela Educação Física;
	Educação Física e suas relações com o mundo do trabalho, a saúde e o lazer;	
	\item \textbf{Bibliografia básica}
	\begin{enumerate}
		\item NNNNN
	\end{enumerate}
	\item \textbf{Bibliografia complementar}
	\begin{enumerate}
		\item 
	\end{enumerate}	
\end{itemize}


\newpage
\subsection{Língua portuguesa, leitura e produção de textos}\label{disc:linguaportuguesa}
\begin{itemize}
	\item \textbf{Carga horária (hora/aula):} 324
	\item \textbf{Docente Responsável:}
	\item \textbf{Conceitos-Chave:}
	\begin{itemize}
		\item NNNNN
		\item NNNNN
	\end{itemize}
	\item \textbf{Ementa 1º Ano:} 
	Práticas de leitura, compreensão, interpretação e produção de textos de diversos gêneros textuais
	Análise linguística: integração dos níveis morfossintático e discursivo;
	Literatura brasileira e seus aspectos estilísticos e culturais em diálogo com a cultura afro-brasileira e indígena;
	Usos da Língua em diferentes registros e níveis de formalidade;

	\item \textbf{Ementa 2º Ano:} 
	Práticas de leitura, compreensão, interpretação e produção de textos de diversos gêneros textuais
	Análise linguística: integração dos níveis morfossintático e discursivo;
	Literatura brasileira e seus aspectos estilísticos e culturais em diálogo com a cultura afro-brasileira e indígena;
	Usos da Língua em diferentes registros e níveis de formalidade;

	\item \textbf{Ementa 3º Ano:} 
	Práticas de leitura, compreensão, interpretação e produção de textos de diversos gêneros textuais
	Análise linguística: integração dos níveis morfossintático e discursivo;
	Literatura brasileira e seus aspectos estilísticos e culturais em diálogo com a cultura afro-brasileira e indígena;
	Usos da Língua em diferentes registros e níveis de formalidade;	
	
	\item \textbf{Bibliografia básica}
	\begin{enumerate}
		\item NNNNN
	\end{enumerate}
	\item \textbf{Bibliografia complementar}
	\begin{enumerate}
		\item 
	\end{enumerate}	
\end{itemize}


\newpage
\subsection{Matemática e estatística}\label{disc:matematica}
\begin{itemize}
	\item \textbf{Carga horária (hora/aula):} 216
	\item \textbf{Docente Responsável:}
	\item \textbf{Conceitos-Chave:}
	\begin{itemize}
		\item NNNNN
		\item NNNNN
	\end{itemize}
	\item \textbf{Ementa 1º Ano:} 
	Conjuntos;
	Funções: introdução, afim, quadrática, modular, exponencial e logarítmica;
	Progressão aritmética;
	Progressão geométrica;
	\item \textbf{Ementa 2º Ano:} 
	Trigonometria;
	Funções trigonométricas;
	Geometria plana e espacial;
	Sistemas lineares;
	Matrizes;
	Determinantes;
	Noções de Estatística e Probabilidade;
	Conceitos básicos de Bioestatística, tais como: organização dos dados quantitativos;
	Medidas de tendência central e de dispersão; distribuições; formulação de testes de hipóteses; 
	Médias e correlações;	
	\item \textbf{Ementa 3º Ano:} 
	Geometria analítica; 
	Equações polinomiais; 
	Números complexos; 
	Combinatória; 

	\item \textbf{Bibliografia básica}
	\begin{enumerate}
		\item NNNNN
	\end{enumerate}
	\item \textbf{Bibliografia complementar}
	\begin{enumerate}
		\item 
	\end{enumerate}	
\end{itemize}


\newpage
\subsection{Geografia}\label{disc:geografia}
\begin{itemize}
	\item \textbf{Carga horária (hora/aula):} 162
	\item \textbf{Docente Responsável:}
	\item \textbf{Conceitos-Chave:}
	\begin{itemize}
		\item NNNNN
		\item NNNNN
	\end{itemize}
	\item \textbf{Ementa 1º Ano:} 
	A contribuição da Geografia para compreensão da realidade/mundo;
	Formas de representação espacial;
	A dinâmica da natureza e as interfaces com a formação das paisagens;
	Apropriação da natureza pelo trabalho e a questão ambiental;
	\item \textbf{Ementa 2º Ano:} 	
	Espacialização das relações capitalistas de produção e a sociedade em rede;
	O processo de urbanização e a questão campo/cidade;
	Dinâmica demográfica e as relações étnico-culturais mundiais;
	Regionalização do espaço mundial e as novas modalidades de exclusão;
	Território, conflitos e geopolítica mundial;
	\item \textbf{Ementa 3º Ano:} 	
	Constituição do território brasileiro;
	Formação das identidades no Brasil; 
	Dinâmica da natureza e a paisagem brasileira;
	Desenvolvimento industrial e urbanização no Brasil;
	Ocupação produtiva e a agricultura no Brasil; 
	Dinâmica demográfica e relações étnico-culturais no Brasil;
	Geografia de Goiás;
	\item \textbf{Bibliografia básica}
	\begin{enumerate}
		\item NNNNN
	\end{enumerate}
	\item \textbf{Bibliografia complementar}
	\begin{enumerate}
		\item 
	\end{enumerate}	
\end{itemize}


\newpage
\subsection{História}\label{disc:historia}
\begin{itemize}
	\item \textbf{Carga horária (hora/aula):} 162
	\item \textbf{Docente Responsável:}
	\item \textbf{Conceitos-Chave:}
	\begin{itemize}
		\item NNNNN
		\item NNNNN
	\end{itemize}
	\item \textbf{Ementa 1º Ano:} 
	Estudos históricos em relações entre trabalho, produção, tecnologia, ciência, meio ambiente, questões étnico-culturais, de gênero, memória e as articulações destes elementos no interior de cada formação social, articulando o global e o local, bem como suas implicações nas diversas realidades; 
	Análise de processos de transformações/permanências/ resistências/semelhanças e diferenças nas dimensões políticas, econômicas, sociais e culturais;
	Sociedades ágrafas, antigas e medievais;
	\item \textbf{Ementa 2º Ano:} 
	Estudos históricos em relações entre trabalho, produção, tecnologia, ciência, meio ambiente, questões étnico-culturais, de gênero, memória e as articulações destes elementos no interior de cada formação social, articulando o global e o local, bem como suas implicações nas diversas realidades; 
    Análise de processos de transformações/permanências/ resistências/semelhanças e diferenças nas dimensões políticas, econômicas, sociais e culturais;		
	Construção do mundo moderno: Europa, Ásia, Áfricas, Américas;
	Processos revolucionários dos séculos XVIII e XIX; 
	Brasil Império;
	\item \textbf{Ementa 3º Ano:} 
	Estudos históricos em relações entre trabalho, produção, tecnologia, ciência, meio ambiente, questões étnico-culturais, de gênero, memória e as articulações destes elementos no interior de cada formação social, articulando o global e o local, bem como suas implicações nas diversas realidades; 
	Análise de processos de transformações/permanências/ resistências/semelhanças e diferenças nas dimensões políticas, econômicas, sociais e culturais;			
	Construção do mundo contemporâneo: do imperialismo à globalização; 
	Brasil República;
	\item \textbf{Bibliografia básica}
	\begin{enumerate}
		\item NNNNN
	\end{enumerate}
	\item \textbf{Bibliografia complementar}
	\begin{enumerate}
		\item 
	\end{enumerate}	
\end{itemize}


\newpage
\subsection{Física}\label{disc:fisica}
\begin{itemize}
	\item \textbf{Carga horária (hora/aula):} 162
	\item \textbf{Docente Responsável:}
	\item \textbf{Conceitos-Chave:}
	\begin{itemize}
		\item NNNNN
		\item NNNNN
	\end{itemize}
	\item \textbf{Ementa 1º Ano:} 
	Movimentos: variações e conservações;
	\item \textbf{Ementa 2º Ano:}	
	Calor, ambiente e uso de energia;
	Som, imagem e informação;
	\item \textbf{Ementa 3º Ano:}	
	Equipamentos elétricos e telecomunicações;
	Matéria e radiação\footnote{ver intersecção com~\nameref{disc:quimica}: Noções de radioatividade;};
	\item \textbf{Bibliografia básica}
	\begin{enumerate}
		\item NNNNN
	\end{enumerate}
	\item \textbf{Bibliografia complementar}
	\begin{enumerate}
		\item 
	\end{enumerate}	
\end{itemize}


\newpage
\subsection{Química}\label{disc:quimica}
\begin{itemize}
	\item \textbf{Carga horária (hora/aula):} 108
	\item \textbf{Docente Responsável:}
	\item \textbf{Conceitos-Chave:}
	\begin{itemize}
		\item NNNNN
		\item NNNNN
	\end{itemize}
	\item \textbf{Ementa 1º Semestre:} 
	%Leis ponderais;
	Matéria, energia, transformações, substâncias;
	Modelos e estrutura atômica;
	Tabela periódica;
	Ligações e interações químicas;
	Funções inorgânicas;
	Reações químicas;
	\item \textbf{Ementa 2º Semestre:} 	
	Noções de radioatividade;
	Equilíbrio em meio homogêneo (Ácido - Base): teoria ácido-base (segundo Arhenius, Brönsted e Lewis);
	Equilíbrio químico;
	Introdução à química orgânica;
	Funções orgânicas: hidrocarbonetos, oxigenadas e nitrogenadas, e suas principais reações; 
	Isomeria;
	%Parte de fisico-quimica ficou junto com quimica analitica
	\item \textbf{Bibliografia básica}
	\begin{enumerate}
		\item NNNNN
	\end{enumerate}
	\item \textbf{Bibliografia complementar}
	\begin{enumerate}
		\item 
	\end{enumerate}	
\end{itemize}


\newpage
\subsection{Biologia}\label{disc:biologia}
\begin{itemize}
	\item \textbf{Carga horária (hora/aula):} 108
	\item \textbf{Docente Responsável:}
	\item \textbf{Conceitos-Chave:}
	\begin{itemize}
		\item NNNNN
		\item NNNNN
	\end{itemize}
	\item \textbf{Ementa 1º Semestre:} 
	Seres vivos: Classificação, organização e importância econômica e ambiental;
	Ciclos Biogeoquímicos; 
	Célula: teoria, padrões e componentes; 
	Divisão celular;
	Morfologia e fisiologia humana;
	
	\item \textbf{Ementa 2º Semestre:} 
	Origem da vida; 
	Teorias e mecanismos evolutivos;
	Ecologia: Conceitos básicos, ecologia de população, comunidades e ecossistemas; 
	Poluição e sustentabilidade;
	
	\item \textbf{Bibliografia básica}
	\begin{enumerate}
		\item NNNNN
	\end{enumerate}
	\item \textbf{Bibliografia complementar}
	\begin{enumerate}
		\item 
	\end{enumerate}	
\end{itemize}

\newpage
\subsection{Filosofia}\label{disc:filosofia}
\begin{itemize}
	\item \textbf{Carga horária (hora/aula):} 162
	\item \textbf{Docente Responsável:}
	\item \textbf{Conceitos-Chave:}
	\begin{itemize}
		\item NNNNN
		\item NNNNN
	\end{itemize}
	\item \textbf{Ementa 1º Ano:} 
	Introdução à filosofia e ao filosofar;
	Elementos conceituais da teoria do conhecimento, da ontologia e das estruturas do pensamento e da linguagem;
	\VER{Abordagem da ética filosófica à ética aplicada em saúde;}\footnote{Veio de~\nameref{disc:bioeticalab}} 
	\item \textbf{Ementa 2º Ano:} 	
	Fundamentos, concepções e relações da ética e da política; 
	Valores, direitos humanos, liberdade e virtude;
	\sout{Estado, poder, soberania, ideologia e formas de governo;}\footnote{Vai para~\nameref{disc:sociologia}}
	\item \textbf{Ementa 3º Ano:} 
	Fundamentos conceituais da ciência, da subjetividade e da estética;
	O significado e as implicações dos processos científicos e da técnica; 
	A crise da razão;
	\item \textbf{Bibliografia básica}
	\begin{enumerate}
		\item NNNNN
	\end{enumerate}
	\item \textbf{Bibliografia complementar}
	\begin{enumerate}
		\item 
	\end{enumerate}	
\end{itemize}


\newpage
\subsection{Sociologia}\label{disc:sociologia}
\begin{itemize}
	\item \textbf{Carga horária (hora/aula):} 162
	\item \textbf{Docente Responsável:}
	\item \textbf{Conceitos-Chave:}
	\begin{itemize}
		\item NNNNN
		\item NNNNN
	\end{itemize}
	\item \textbf{Ementa 1º Ano:} 
	A Sociologia como ciência e sua origem; 
	Indivíduo e sociedade; 
	Instituições sociais; 
	Correntes clássicas do pensamento sociológico; 
	Modernidade e capitalismo;
	\item \textbf{Ementa 2º Ano:} 
	Cultura, etnocentrismo, relativismo cultural e diversidade: relações étnico-raciais, gênero, geração, sexualidade;
	Educação e sociedade; 
	Desigualdades sociais; 
	Trabalho e organização produtiva; 
	Globalização e Mundialização do do capital; 
	Indústria cultural e consumo;
	\item \textbf{Ementa 3º Ano:} 
	\VER{	
	Estado, ideologia e regimes políticos; 
	Sistemas de governo;
	}\footnote{Veio de~\nameref{disc:filosofia}} 
	Movimentos sociais;
	Cidadania e participação social;
	Política;
	\item \textbf{Bibliografia básica}
	\begin{enumerate}
		\item NNNNN
	\end{enumerate}
	\item \textbf{Bibliografia complementar}
	\begin{enumerate}
		\item 
	\end{enumerate}	
\end{itemize}


\newpage
\section{Ementas do Núcleo Politécnico}\label{ementasPolitecnico}
\indent

O núcleo politécnico compreende os fundamentos científicos, sociais, organizacionais, econômicos, políticos, culturais, ambientais, estéticos e éticos que alicerçam as tecnologias e a contextualização do mesmo no sistema de produção social de forma aderenta ao eixo tecnológico em que se situa o curso~\cite{Resolucao06De2012}.

\newpage
\subsection{Inovação e Tecnlogia da Informação}\label{disc:info}
\begin{itemize}
	\item \textbf{Carga horária (hora/aula):} 54
	\item \textbf{Docente Responsável:}
	\item \textbf{Conceitos-Chave:}
	\begin{itemize}
		\item Tecnologia da informação
		\item Metodologia científica
		\item Propriedade intelectual
		\item Gestão
		\item Inovação
		\item Empreendedorismo
	\end{itemize}
	\item \textbf{Ementa:} 
	Uso da Internet e Noções de segurança da informação;
	Produção de textos usando software;
	Produção de planilha usando software;
	Produção de apresentações usando software;
	Elaboração de projetos de pesquisa; 
	Estrutura do trabalho científico;
	Propriedade intelectual: conceitos e modalidades;
	Gestão da propriedade intelectual;
	Gestão da inovação e transferência de tecnologia;
	Prospecção tecnológica;
	Noções de empreendedorismo;
	\item \textbf{Bibliografia básica}
	\begin{enumerate}
		\item NNNNN
	\end{enumerate}
	\item \textbf{Bibliografia complementar}
	\begin{enumerate}
		\item 
	\end{enumerate}	
\end{itemize}

\newpage
\subsection{Língua opcional – Espanhol ou LIBRAS}\label{disc:espanhol_libras}
\begin{itemize}
	\item \textbf{Carga horária (hora/aula):} 54
	\item \textbf{Docente Responsável:}
	\item \textbf{Conceitos-Chave:}
	\begin{itemize}
		\item NNNNN
		\item NNNNN
	\end{itemize}
	\item \textbf{Ementa:} 
	Estruturas básicas da Língua Espanhola em uma abordagem contrastiva com a Língua Portuguesa em seus aspectos lexicais, sintáticos, semânticos, pragmáticos, discursivos e interculturais; 
	Habilidades comunicativas de recepção e produção em vários gêneros textuais a partir das especificidades de cada curso;
	Aspectos histórico-culturais do surdo. Noções básicas da gramática da Língua Brasileira de Sinais (LIBRAS);
	Vocabulário básico da LIBRAS;
	Práticas de conversação em LIBRAS;		
	\item \textbf{Bibliografia básica}
	\begin{enumerate}
		\item NNNNN
	\end{enumerate}
	\item \textbf{Bibliografia complementar}
	\begin{enumerate}
		\item 
	\end{enumerate}	
\end{itemize}


\newpage
\subsection{Fermentação}\label{disc:fermentacao}
\begin{itemize}
	\item \textbf{Carga horária (hora/aula):} 54
	\item \textbf{Docente Responsável:}
	\item \textbf{Conceitos-Chave:}
	\begin{itemize}
		\item NNNNN
		\item NNNNN
	\end{itemize}
	\item \textbf{Ementa:} 
	Fermentação industriais;
	Conceituação de processo fermentativo; 
	Microrganismos para utilização industrial; 
	Matérias primas e meios de fermentação para utilização industrial; 
	Principais etapas de um processo fermentativo; 
	Classificação dos processos fermentativos quanto ao desenvolvimento do agente, regime de condução do processo e necessidade de oxigênio;
	Produtos do metabolismo microbiano de interesse na indústria farmacêutica, de alimentos e afins; 
	Enzimologia industrial; 
	Cinética de crescimento microbiano;
	Esterilização de equipamentos, meios e ar;
	Biorreatores;
	Bioprocessos.;
	\item \textbf{Bibliografia básica}
	\begin{enumerate}
		\item NNNNN
	\end{enumerate}
	\item \textbf{Bibliografia complementar}
	\begin{enumerate}
		\item 
	\end{enumerate}	
\end{itemize}

\newpage
\subsection{Bioquímica}\label{disc:bioquimica}
\begin{itemize}
	\item \textbf{Carga horária (hora/aula):} 108
	\item \textbf{Docente Responsável:}
	\item \textbf{Conceitos-Chave:}
	\begin{itemize}
		\item NNNNN
		\item NNNNN
	\end{itemize}
	\item \textbf{Ementa:} 
	Introdução à Bioquímica; 
	Biomoléculas e nutrientes;
	Reações de biossíntese e degradação;
	Metabolismo e aplicações de carboidratos, lipídios e proteínas;
	Seminários de bioquímica;
	\item \textbf{Bibliografia básica}
	\begin{enumerate}
		\item NNNNN
	\end{enumerate}
	\item \textbf{Bibliografia complementar}
	\begin{enumerate}
		\item 
	\end{enumerate}	
\end{itemize}

\newpage
\subsection{Microbiologia}\label{disc:microbiologia}
\begin{itemize}
	\item \textbf{Carga horária (hora/aula):} 108
	\item \textbf{Docente Responsável:}
	\item \textbf{Conceitos-Chave:}
	\begin{itemize}
		\item NNNNN
		\item NNNNN
	\end{itemize}
	\item \textbf{Ementa:}
    Introdução e histórico da microbiologia; 
	Microrganismos: classificação, citologia, morfologia, metabolismo, crescimento, controle do crescimento, genética e técnicas microbiológicas;
	Microbiologia industrial; 
	Principais microrganismos e bioprodutos industriais: produção, melhoramento e características gerais;
	
	\item \textbf{Bibliografia básica}
	\begin{enumerate}
		\item NNNNN
	\end{enumerate}
	\item \textbf{Bibliografia complementar}
	\begin{enumerate}
		\item 
	\end{enumerate}	
\end{itemize}

\newpage
\subsection{Biologia Molecular e Bioinformática}\label{disc:biomol}
\begin{itemize}
	\item \textbf{Carga horária (hora/aula):} 108
	\item \textbf{Docente Responsável:}
	\item \textbf{Conceitos-Chave:}
	\begin{itemize}
		\item NNNNN
		\item NNNNN
	\end{itemize}

	\item \textbf{Ementa 1º semestre:}
    Dogma central da Biologia Molecular e o fluxo da informação genética;
	Estrutura, propriedades e características de ácidos nucléicos (DNA e RNA);
	Código genético; 
	Replicação e transcrição e tradução em procariotos e eucariotos;
	Mecanismo de processamento do mRNA eucariótico; 
	Histonas e empacotamento do DNA eucariótico; 
	Biossíntese de proteínas; 
	Amplificação gênica \textit{in vivo} e \textit{in vitro}; 
	Reparo e mutagênese;
	Técnicas básicas de manipulação genética;
	
	\item \textbf{Ementa 2º semestre:}	
	Genômica;
	Transcritômica;
	Metabolômica;
	Bioinformática básica;
	Noções de programação de computadores;
	Bancos de dados biológicas;
	Montagem e anotação de genomas;
	Análises de RNA-Seq;	
	\item \textbf{Bibliografia básica}
	\begin{enumerate}
		\item NNNNN
	\end{enumerate}
	\item \textbf{Bibliografia complementar}
	\begin{enumerate}
		\item 
	\end{enumerate}	
\end{itemize}



%\newpage
%\subsection{Bioética}\label{disc:bioetica}
%\begin{itemize}
%	\item \textbf{Carga horária (hora/aula):} 108
%	\item \textbf{Docente Responsável:}
%	\item \textbf{Conceitos-Chave:}
%	\begin{itemize}
%		\item NNNNN
%		\item NNNNN
%	\end{itemize}
%	\item \textbf{Ementa:} 
%	Abordagem da ética filosófica à ética aplicada em saúde; princípios e teorias da bioética;
%	Produção de conhecimento e o exercício profissional em biotecnologia; 
%	Papel e limites das ciências e do cientista; 
%	Discussão de questões teóricas voltadas a questões da bioética constitutivas dos campos das relações emergentes e das relações persistentes de nossa sociedade; 
%	Bioética e a saúde pública, eutanásia e distanásia, segurança alimentar; 
%	Transgênicos; 
%	Especismo; 
%	tecnologias de ponta, bioterrorismo; 
%	Aborto;
%	\item \textbf{Bibliografia básica}
%	\begin{enumerate}
%		\item NNNNN
%	\end{enumerate}
%	\item \textbf{Bibliografia complementar}
%	\begin{enumerate}
%		\item 
%	\end{enumerate}	
%\end{itemize}


\newpage
\subsection{Bioética, biossegurança e fundamentos de laboratório}\label{disc:bioeticalab}
\begin{itemize}
	\item \textbf{Carga horária (hora/aula):} 108
	\item \textbf{Docente Responsável:}
	\item \textbf{Conceitos-Chave:}
	\begin{itemize}
		\item NNNNN
		\item NNNNN
	\end{itemize}
	\item \textbf{Ementa 1º Semestre:} 
	\sout{Abordagem da ética filosófica à ética aplicada em saúde;}\footnote{Foi para~\nameref{disc:filosofia}}
	Princípios e teorias da bioética;
	Produção de conhecimento e o exercício profissional em biotecnologia; 
	Papel e limites das ciências e do cientista; 
	Discussão de questões teóricas voltadas a questões da bioética constitutivas dos campos das relações emergentes e das relações persistentes de nossa sociedade; 
	Bioética e a saúde pública, eutanásia e distanásia, segurança alimentar; 
	Transgênicos; 
	Especismo; 
	Tecnologias de ponta, bioterrorismo; 
	Aborto;
	Direitos humanos;
	\item \textbf{Ementa 2º Semestre:} 
	Conceito de Biossegurança e importância; 
	Legislação, normas e medidas de biossegurança nas atividades desenvolvidas pelos profissionais de biotecnologia; 
	Riscos químicos, físicos e biológicos; 
	Condutas de segurança e saúde no trabalho; 
	Transporte e descarte dos resíduos de serviço de saúde e relação com o meio ambiente;
	\item \textbf{Bibliografia básica}
	\begin{enumerate}
		\item NNNNN
	\end{enumerate}
	\item \textbf{Bibliografia complementar}
	\begin{enumerate}
		\item 
	\end{enumerate}	
\end{itemize}


\newpage
\section{Ementas do Núcleo Profissional}\label{ementasTecnico}
\indent

O núcleo profissional contempla métodos, técnicas, ferramentas e outros elementos das tecnologias relativas aos cursos~\cite{Resolucao06De2012}.

\newpage
\subsection{Biotecnologia animal}\label{disc:biotecAnimal}
\begin{itemize}
	\item \textbf{Carga horária (hora/aula):} 54
	\item \textbf{Docente Responsável:}
	\item \textbf{Conceitos-Chave:}
	\begin{itemize}
		\item NNNNN
		\item NNNNN
	\end{itemize}
	\item \textbf{Ementa:} 
	Zoologia: classificação, organização e fisiologia;
	Fundamentos de regulação homestática, nutrição, digestão, metabolismo, o smorregulação e excreção, ventilação e circulação, músculo e movimento, regulação neuroendócrina, reprodução, Coordenação e interação dos organismos animais, evolução e filogênese do sistema nervoso; 
	Sistema sensorial e motor de invertebrados e vertebrados; 
	Técnicas de controle de pragas \textit{in vivo} e \textit{in vitro};
	Biotecnologia Animal no Brasil e no mundo; 
	Situação atual e perspectivas.
	\item \textbf{Bibliografia básica}
	\begin{enumerate}
		\item NNNNN
	\end{enumerate}
	\item \textbf{Bibliografia complementar}
	\begin{enumerate}
		\item 
	\end{enumerate}	
\end{itemize}


\newpage
\subsection{Biotecnologia vegetal}\label{disc:biotecVegetal}
\begin{itemize}
	\item \textbf{Carga horária (hora/aula):} 54
	\item \textbf{Docente Responsável:}
	\item \textbf{Conceitos-Chave:}
	\begin{itemize}
		\item NNNNN
		\item NNNNN
	\end{itemize}
	\item \textbf{Ementa:} 
	Botânica: Classificação, Organização e Fisiologia; 
	Anatomia dos órgãos vegetativos e reprodutivos;
	Estruturas, primária e secundária, das raízes e dos caules; 
	Estrutura básica e desenvolvimento da folha; 
	Variações estruturais da folha relacionadas com o hábitat;
	Respiração; 
	Fotossíntese;
	O fluxo de energia nas plantas; 
	Protistas fotossintetizantes, briófitas, plantas vasculares sem sementes, gimnospermas e divisão anthophyta: tecidos simples e complexos; 
	Hormônios Vegetais; 
	Fatores externos e crescimento vegetal; 
	Nutricão vegetal e solos; 
	O movimento da água e solutos nas plantas; 
	\item \textbf{Bibliografia básica}
	\begin{enumerate}
		\item NNNNN
	\end{enumerate}
	\item \textbf{Bibliografia complementar}
	\begin{enumerate}
		\item 
	\end{enumerate}	
\end{itemize}


\newpage
\subsection{Biotecnologia de alimentos}\label{disc:biotecAlimentos}
\begin{itemize}
	\item \textbf{Carga horária (hora/aula):} 108
	\item \textbf{Docente Responsável:}
	\item \textbf{Conceitos-Chave:}
	\begin{itemize}
		\item NNNNN
		\item NNNNN
	\end{itemize}
	\item \textbf{Ementa:} 
	Introdução aos princípios e processos tecnológicos envolvidos no processamento de alimentos;
	Estudos das modificações bioquímicas dos alimentos durante o desenvolvimento, armazenamento e processamento;
	Fundamentos da produção biotecnológica para o desenvolvimento de produtos e processos alimentícios (carnes, laticínios, cereais vegetais, ovo, pães, aditivos e derivados);
	Boas práticas de manufatura;
	Análise de risco e pontos críticos de controle;
	Conservação de alimentos;
	Embalagens;
	Bioquímica e bromatologia de alimentos;
	\item \textbf{Bibliografia básica}
	\begin{enumerate}
		\item NNNNN
	\end{enumerate}
	\item \textbf{Bibliografia complementar}
	\begin{enumerate}
		\item 
	\end{enumerate}	
\end{itemize}


\newpage
\subsection{Biotecnologia de fármacos e biodefensivos}\label{disc:biotecFarmacos}
\begin{itemize}
	\item \textbf{Carga horária (hora/aula):} 108
	\item \textbf{Docente Responsável:}
	\item \textbf{Conceitos-Chave:}
	\begin{itemize}
		\item NNNNN
		\item NNNNN
	\end{itemize}
	\item \textbf{Ementa:} 
	Pesquisa e Produção de biofármacos e biodefensivos em escala laboratorial e industrial;
	Vacinas, antibióticos, antifúngicos, fatores sanguíneos, hormônios, interferons, interleucinas,
	anticorpos monoclonais, enzimas; 
	Fármacos de origem natural;
	\item \textbf{Bibliografia básica}
	\begin{enumerate}
		\item NNNNN
	\end{enumerate}
	\item \textbf{Bibliografia complementar}
	\begin{enumerate}
		\item 
	\end{enumerate}	
\end{itemize}


\newpage
\subsection{Biotecnologia humana e saúde}\label{disc:biotecSaude}
\begin{itemize}
	\item \textbf{Carga horária (hora/aula):} 108
	\item \textbf{Docente Responsável:}
	\item \textbf{Conceitos-Chave:}
	\begin{itemize}
		\item NNNNN
		\item NNNNN
	\end{itemize}
	\item \textbf{Ementa:} 
	Introdução à Genética;
	Probabilidade e teste de proporções genéticas; 
	Mendelismo: os princípios básicos da herança; 
	Extensões do mendelismo; 
	Genes ligados ao sexo em seres humanos;
	Genética quantitativa: modelos para cor da pele humana e discussão das questões étnico-raciais à luz da genética moderna;
	Variação no número e estrutura dos cromossomos;
	Relação dos parasitos e seus efeitos no sistema imune do hospedeiro; 
	Identificação dos parasitos que acometem o homem e alguns os animais domésticos: protozoologia, helmintologia, entomologia e acarologia, as formas de transmissão e diagnósticos laboratoriais; 
	Epidemiologia e profilaxia; 
	Estudo dos mecanismos da resposta imune inata e adaptativa; 
	Reconhecimento de antígenos; 
	Maturação, ativação e regulação dos linfócitos;
	Mecanismos efetores envolvidos na resposta imune;
	Processos patológicos decorrentes de alterações nos mecanismos normais de resposta imunológica;
	\item \textbf{Bibliografia básica}
	\begin{enumerate}
		\item NNNNN
	\end{enumerate}
	\item \textbf{Bibliografia complementar}
	\begin{enumerate}
		\item 
	\end{enumerate}	
\end{itemize}


\newpage
\subsection{Produção de bioprodutos}\label{disc:producao}
\begin{itemize}
	\item \textbf{Carga horária (hora/aula):} 108
	\item \textbf{Docente Responsável:}
	\item \textbf{Conceitos-Chave:}
	\begin{itemize}
		\item NNNNN
		\item NNNNN
	\end{itemize}
	\item \textbf{Ementa:} 
	Técnicas e metodologias de extração e purificação: extração líquido-líquido, extração em fase sólida, extração com fluido supercrítico e extração com membranas sólidas (diálise e ultrafiltração) ou líquidas, infusão, decocção, percolação, teoria do soxhlet, arraste por vapor d’água, turbólize, maceração e variáveis, ultrassom, agitação mecânica, cristalização, centrifugação, adsorção, dissolução, filtração, concentração, liofilização; 
	Técnicas e metodologias de separação: cromatografia, eletroforese: tipos, definições carcaterísticas gerais, procedimentos, exemplos. 
	Técnicas e metodologias de identificação de compostos orgânicos: ressonância magnética nuclear, espectroscopia no infravermelho, ultra-violeta e visível e espectrometria de massas; 
	Aulas práticas de extração, separação e identificação e substâncias;
	Introdução ao controle de qualidade; 
	Ferramentas de qualidade; 
	Sistemas e gestão da qualidade;
	Noções de bioeconomia;
	\item \textbf{Bibliografia básica}
	\begin{enumerate}
		\item NNNNN
	\end{enumerate}
	\item \textbf{Bibliografia complementar}
	\begin{enumerate}
		\item 
	\end{enumerate}	
\end{itemize}



\newpage
\subsection{Físico Química e Química Analítica}\label{disc:analitica}
\begin{itemize}
	\item \textbf{Carga horária (hora/aula):} 54
	\item \textbf{Docente Responsável:}
	\item \textbf{Conceitos-Chave:}
	\begin{itemize}
		\item NNNNN
		\item NNNNN
	\end{itemize}
	\item \textbf{Ementa:} 
	%Reações de complexação; 	
	%Marcha geral de análise; 	
	%Volumetria de complexação;
	Estequiometria;
	Soluções e propriedades coligativas;
	Eletroquímica;
	Termoquímica;
	Cinética química;
	Introdução ao Estudo de Química Analítica: marcha geral de análise, seletividade e especificidade, sensibilidade ou limite de detecção;
	Reações Redox; 
	Método gráfico para determinação e especiação das espécies químicas estudadas;
	Análise sistemática x Análise assistemática: análise de cátions; 
	Métodos quantitativos;
	Amostragem; 
	Medição em química analítica; 
	Material volumétrico e balança analítica; 
	Introdução à análise volumétrica;
	Volumetria de neutralização;
	Análise gravimétrica; 
	Volumetria de oxidação-redução; 
	Volumetria de precipitação; 
	Potenciometria; 
	Absorção atômica;
	\item \textbf{Bibliografia básica}
	\begin{enumerate}
		\item NNNNN
	\end{enumerate}
	\item \textbf{Bibliografia complementar}
	\begin{enumerate}
		\item 
	\end{enumerate}	
\end{itemize}

\newpage
\section{Estudo orientado e debates}\label{ementasEstudo}
\indent

Esta é uma disciplina de ementa aberta que visa assegurar tempos e espaços para que os estudantes reflitam sobre suas experiências e aprendizagens individuais e interpessoais, de modo a valorizarem o conhecimento, confiarem em sua capacidade de aprender, e identificarem e utilizarem estratégias mais eficientes a seu aprendizado~\cite{BNCC2019}.

Alicerçados no conhecimento e na inovação, e promovendo a aprendizagem colaborativa, os estudos orientado e debates desenvolvem nos estudantes a capacidade de trabalharem em equipe e aprenderemcom seus pares, além de estimular atitudes cooperativas e propositivas preparando-os para os desafios da comunidade, do mundo do trabalho e da sociedade em geral.
Desta forma, o objetivo da disciplina atendo à meta da BNCC garantindo a contextualização dos conhecimentos, articulando as dimensões do trabalho, da ciência, da tecnologia e da cultura~\cite{BNCC2019}.


\subsection{Estudo orientado e debates}\label{disc:estudo}
\begin{itemize}
	\item \textbf{Carga horária (hora/aula):} 324
	\item \textbf{Docente Responsável:}
	\item \textbf{Conceitos-Chave:}
	\begin{itemize}
		\item NNNNN
		\item NNNNN
	\end{itemize}
	\item \textbf{Ementa:} Momento presencial com carga horária de 8 horas por semana para estudos dirigidos interdisciplinares; Debates de temas com dois grupos: um pró e um contra;
	\item \textbf{Bibliografia básica}
	\begin{enumerate}
		\item NNNNN
	\end{enumerate}
	\item \textbf{Bibliografia complementar}
	\begin{enumerate}
		\item 
	\end{enumerate}	
\end{itemize}


%----------------------------------------------------------------------------------------
%	CHAPTER X
%----------------------------------------------------------------------------------------
%------------------------------------------------
\chapterimage{05.jpg} % Chapter heading image

\chapter{Estrutura Física}
\vspace{6em}
\begin{flushright}
	\textit{\textcolor{white}{Foto: Adriano Darosci}}
\end{flushright}
\vspace{12em}

\section{Laboratório de Fisiologia Vegetal}

Equipado com: estufa de secagem, 3 estereoscópios, 3 microscópicos, geladeira, bancadas, 28 cadeiras, quadro e acervo didático (frutos, sementes e folhas herborizadas). 

\section{Laboratório de Bioqímica}

Equipado com: Balanças analítica e semi-analítica, chapas de aquecimento (com agitação magnética), analisador bioquímico, capela de fluxo laminar, agitadores de tubo de ensaio, banho-maria, bomba de vácuo, autoclave, estufas, destilador e deionizador de água e outros.

\section{Laboratório de Anatomia e Zoologia}

Equipado com: Bonecos anatômicos (de abdome) completos, conjuntos anatômicos artificiais de sistemas reprodutores femininos e masculinos, esqueletos completos (artificiais), amostras de animais (do cerrado e de outros biomas) conservados em frascos para visualização, animais empalhados, algumas peças anatômicas naturais de animais, lupas, microscópios, material para coleta de animais e saídas de campo, materiais e reagentes para o empalhamento de animais e outros.

\section{Laboratório de Microscopia e Microbiologia}

Equipado com:  25 microscópios e material para produção de lâminas (lâminas de corte, lâminas e lamínulas de vidro, corantes, fixadores, etc); Lupas, coleções de laminários e outros.

\section{Laboratório de Físico-Química}

Equipado com: pHmetros, destilador, capela de exaustão, estufa, banho-maria, balanças analítica e semi-analítica, deionizador, reator, aparelho de ponto de fusão,  e outros.

\section{Laboratório de Águas Residuais}

Equipado com: Condutivímetros, muflas, banho - maria, bomba de vácuo, analisador de oxigênio dissolvido, turbdímetro, estufa, balança, phmetro, destilador e outros.

\section{Laboratório de Ensino}

Espaço acadêmico voltado ao desenvolvimento e disseminação de tecnologias educacionais voltadas ao ensino de Ciências e Biologia.  Equipado com: acervo didático constituído por jogos, maquetes e representações físicas de organismos e processos biológicos.

\section{Laboratório de Física e Matemática}

O Laboratório de Física possui diversos equipamentos que contribui para o desenvolvimento das atividades experimentais nas áreas de mecânica, óptica, hidrostática, termologia e eletricidade.


\section{Laboratórios de Informática}

Dois laboratórios de informática com capacidade para até 30 estudantes, com acesso à Internet, computadores com sistema operacional Linux, softwares diversos.


\section{Biblioteca}

Biblioteca equipada com áreas de estudo individual e coletivo, 6 computadores com acesso ao portal de periódicos e acervo cerca de 7 mil exemplares, entre livros, livros em braile, cds, dvds e mapas;

\section{Teatro}

Teatro equipado com som e iluminação específica e acomodações para 320 pessoas sentadas;

\section{Outros Espaços}

3 salas para estudos coletivos e reuniões equipadas com mesas, cadeiras e televisor.


%----------------------------------------------------------------------------------------
%	CHAPTER X
%----------------------------------------------------------------------------------------
%------------------------------------------------
\chapterimage{CalliphloxAmethystina.jpg} % Chapter heading image
\chapter{Corpo Docente}
\vspace{6em}
\begin{flushright}
	\textit{\textcolor{white}{Foto: Adriano Darosci}}
\end{flushright}
\vspace{12em}

\section{Adriano Antonio Brito Darosci}\label{AdrianoDarosci}
\begin{itemize}
	\item Formação Básica: Ciências Biológicas
	\item Titulação Máxima: Doutor em Botânica
	\item Regime de Trabalho: Deicação Exclusiva
	\item \includegraphics[scale=.03]{Pictures/lattes}~\href{http://lattes.cnpq.br/4539795481921184}{Lattes: http://lattes.cnpq.br/4539795481921184}
\end{itemize}


\section{Anderson dos Anjos Pereira Pena}\label{AndersonPena}
\begin{itemize}
	\item Formação Básica: Pedagogia
	\item Titulação Máxima: Mestre em Cultura, Memória e Desenvolvimento Regional
	\item Regime de Trabalho: Dedicação Exclusiva
	\item \includegraphics[scale=.03]{Pictures/lattes}~\href{http://lattes.cnpq.br/9188378802285261}{Lattes: http://lattes.cnpq.br/9188378802285261}
\end{itemize}

\section{Ariane Bocaletto Frare}\label{ArianeFrare}
\begin{itemize}
	\item Formação Básica: Ciências Biológicas
	\item Titulação Máxima: Mestre em Genética
	\item Regime de Trabalho: Dedicação Exclusiva
	\item \includegraphics[scale=.03]{Pictures/lattes}~\href{http://lattes.cnpq.br/9984435027737343}{Lattes: http://lattes.cnpq.br/9984435027737343}
\end{itemize}


\section{Daniela Pereira Versieux}\label{DanielaVersieux}
\begin{itemize}
	\item Formação Básica: Ciências Biológicas
	\item Titulação Máxima: Mestre em Educação Tecnológica
	\item Regime de Trabalho: Dedicação Exclusiva
	\item \includegraphics[scale=.03]{Pictures/lattes}~\href{http://lattes.cnpq.br/9970651709122352}{Lattes: http://lattes.cnpq.br/9970651709122352}
\end{itemize}

\section{Fernanda Melo Duarte}\label{FernandaDuarte}
\begin{itemize}
	\item Formação Básica: Ciências Biológicas
	\item Titulação Máxima: Mestre em Genética
	\item Regime de Trabalho: Dedicação Exclusiva
	\item \includegraphics[scale=.03]{Pictures/lattes}~\href{http://lattes.cnpq.br/5338539796531801}{Lattes: http://lattes.cnpq.br/5338539796531801}
\end{itemize}

\section{Haissa Melo de Lima Gunther}\label{HaissaGunther}
\begin{itemize}
	\item Formação Básica: Ciências Biológicas
	\item Titulação Máxima: Mestre em Desenvolvimento Regional e Meio Ambiente
	\item Regime de Trabalho: Dedicação Exclusiva
	\item \includegraphics[scale=.03]{Pictures/lattes}~\href{http://lattes.cnpq.br/8481012955941397}{Lattes: http://lattes.cnpq.br/8481012955941397}
\end{itemize}

\section{Leandro Santos Goulart}\label{LeandroGoulart}
\begin{itemize}
	\item Formação Básica: Ciências Biológicas
	\item Titulação Máxima: Mestre em Biologia Animal
	\item Regime de Trabalho: Dedicação Exclusiva
	\item \includegraphics[scale=.03]{Pictures/lattes}~\href{http://lattes.cnpq.br/1871654436997150}{Lattes: http://lattes.cnpq.br/1871654436997150}
\end{itemize}

\section{Lemuel da Cruz Gandara}\label{LemuelGandara}
\begin{itemize}
	\item Formação Básica: Língua portuguesa e Estudos literários
	\item Titulação Máxima: Doutor em Literatura
	\item Regime de Trabalho: Dedicação Exclusiva
	\item \includegraphics[scale=.03]{Pictures/lattes}~\href{http://lattes.cnpq.br/7649361942295698}{Lattes: http://lattes.cnpq.br/7649361942295698}
\end{itemize}

\section{Marcos Augusto Schliewe}\label{MarcosSchliewe}
\begin{itemize}
	\item Formação Básica: Ciências Biológicas
	\item Titulação Máxima: Doutor em Botânica
	\item Regime de Trabalho: Dedicação Exclusiva
	\item \includegraphics[scale=.03]{Pictures/lattes}~\href{http://lattes.cnpq.br/8055970128960356}{Lattes: http://lattes.cnpq.br/8055970128960356}
\end{itemize}

\section{Patricia de Castilhos}\label{PatriciaCastilhos}
\begin{itemize}
	\item Formação Básica: Ciências Biológicas
	\item Titulação Máxima: Doutora em Imunologia e Parasitologia Aplicadas
	\item Regime de Trabalho: Dedicação Exclusiva
	\item \includegraphics[scale=.03]{Pictures/lattes}~\href{http://lattes.cnpq.br/7391339023174244}{Lattes: http://lattes.cnpq.br/7391339023174244}
\end{itemize}

\section{Thaís Amaral e Sousa}\label{ThaísSousa}
\begin{itemize}
	\item Formação Básica: Ciências Biológicas
	\item Titulação Máxima: Doutora em Ciências Biológicas
	\item Regime de Trabalho: Dedicação Exclusiva
	\item \includegraphics[scale=.03]{Pictures/lattes}~\href{http://lattes.cnpq.br/5246897777497752}{Lattes: http://lattes.cnpq.br/5246897777497752}
\end{itemize}

\section{Vinicius Sousa Ferreira}\label{ViniciusFerreira}
\begin{itemize}
	\item Formação Básica: Química, Farmácia e Bioquímica
	\item Titulação Máxima: Doutor em Química
	\item Regime de Trabalho: Dedicação Exclusiva
	\item \includegraphics[scale=.03]{Pictures/lattes}~\href{http://lattes.cnpq.br/6567799449480628}{Lattes: http://lattes.cnpq.br/6567799449480628}
\end{itemize}

\section{Waldeyr Mendes Cordeiro da Silva}\label{WaldeyrMendes}
\begin{itemize}
	\item Formação Básica: Sistemas de Informação e Ciências Biológicas
	\item Titulação Máxima: Doutor em Ciências Biológicas
	\item Regime de Trabalho: Dedicação Exclusiva
	\item \includegraphics[scale=.03]{Pictures/lattes}~\href{http://lattes.cnpq.br/2391349697609978}{Lattes: http://lattes.cnpq.br/2391349697609978}
	\item \includegraphics[scale=.15]{Pictures/orcid}~\href{https://orcid.org/0000-0002-8660-6331}{ORCID: https://orcid.org/0000-0002-8660-6331}
\end{itemize}
%----------------------------------------------------------------------------------------
%	CHAPTER X
%----------------------------------------------------------------------------------------
%------------------------------------------------
\chapterimage{Hypsiboas.jpg} % Chapter heading image
\chapter*{Conheça o IFG}
\vspace{6em}
\begin{flushright}
%	\textit{{Foto: Adriano Darosci}}
\end{flushright}
\vspace{12em}

\section{Contato}

Instituto Federal de Educação, Ciência e Tecnologia de Goiás - Câmpus Formosa\\
Site:~\url{http://ifg.edu.br}\\
Endereço: XXXXX\\
Telefone: XXXXX \\
Twitter:XXXXXX \\
E-mails: XXXXXXX





% ----------------------------------------------------------------------------------------
% 	BIBLIOGRAPHY
% ----------------------------------------------------------------------------------------
%----------------------------------------------------------------------------------------
%	CHAPTER X
%----------------------------------------------------------------------------------------
%------------------------------------------------
\chapterimage{RoureaInduta.jpg} % Chapter heading image
%\chapter*{Referências Bibliográficas}
%\bibliography{bibliography}
%\renewcommand\bibname{Referências Bibliográficas}

\chapter*{Referências Bibliográficas}
\vspace{6em}
\begin{flushright}
	\textit{\textcolor{white}{Foto: Adriano Darosci}}
\end{flushright}
\vspace{12em}
%\addcontentsline{toc}{chapter}{\textcolor{verde}{Bibliography}}
%\section{Books}
%\addcontentsline{toc}{section}{Books}
%\printbibliography[heading=bibempty,type=book]
%\section{Articles}
%\addcontentsline{toc}{section}{Articles}
%\printbibliography[heading=bibempty,type=article]
\printbibliography[heading=bibempty]


%----------------------------------------------------------------------------------------

%----------------------------------------------------------------------------------------
%	INDEX
%----------------------------------------------------------------------------------------
%
%\cleardoublepage
%\phantomsection
%\setlength{\columnsep}{0.75cm}
%\addcontentsline{toc}{chapter}{\textcolor{verde}{Index}}
%\printindex


\end{document}